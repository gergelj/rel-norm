\chapter{Összefoglalás}

Ebben a dolgozatban bemutatásra került a \textit{RelNorm} nevezetű szoftver, amely relációs adatbázisok normalizálásának a problémáját próbálja megoldani. A \textit{RelNorm} eszközt elsősorban oktatásban ajánlott alkalmazni, normalizálási feladatsorok összeállításánál és a megoldott feladatok leellenőrzésénél. A dolgozat bemutatja a hasonló normalizálási szoftvereket, majd a legalapvetőbb relációs fogalmakra és algoritmusokra tér rá. A \textit{RelNorm} fejlesztési fázisai a(z) \ref{ch:gyakorlat}. fejezetben kerültek be a dolgozatba, ahol kódrészletekkel valamint osztály- és szekvenciadiagramokkal prezentáltuk a szoftver megvalósítását. A \textit{RelNorm} szoftvertesztelésnek vetettük alá, ezek eredményei és kiértékelése a dolgozat utolsó fejezetében történt.

A bemutatott eredmények és azok elemzésének fényében következtethető, hogy a \textit{RelNorm} teljesítette a dolgozatban elé támasztott követelményeket. A gyakorlatban a 2021--2022-es tanévben már helytállt szoftver kielégítette azokat az igényeket, melyeket tanársegédi feladatokként szántunk a szoftverhez.

Jövőbeli terveink között szerepel egy grafikus felhasználói felület kifejlesztése, ami akár web applikációs formát is ölthet: így a \textit{RelNorm} szélesebb rétegekhez is eljuthat. Amennyiben igény van rá, akkor további funkciókkal is bővíthető a program, ami egy tanulási keretet is adhat a hallgatóknak.