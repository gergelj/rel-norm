\chapter{Bevezető}

A relációs adatbázisok alapjául szolgáló relációs adatmodell már az 1970-es években sok tudományos munka tárgyát képezte, és az 1990-es évektől kezdve kezdték alkalmazni kommerciális környezetben is \parencite{mogin1996}. Napjainkban a legkülönfélébb vállalkozások is nagy arányban használnak relációs adatbázisokat valamilyen formában, ezt több felmérés [\parencite{scalegrid2019}, \parencite{ramel2015}] és jelentés \parencite{loukides2022} is bizonyítja. Egy nagy előnye a relációs adatbázisoknak, hogy a relációs adatmodell szilárd matematikai alapokra épül. Ezek az alapok többek között a halmazelméletet és a matematikai relációkat foglalják magukban, ami engedélyezi a relációs algebra és kalkulus műveleteinek a használatát \parencite{mogin1996}.

Annak érdekében, hogy egy relációs adatbázison el tudjunk végezni bizonyos relációs műveleteket, és pontos eredményekhez jussunk, elengedhetetlen az a feltétel, hogy az adatainkat veszteségmentesen egyesíthető relációkba szervezzük. Veszteséges egyesítésnél adatok tűnhetnek el, vagy megjelenhetnek további téves adatok. Ezen probléma kiküszöböléséhez adatbázis normalizálást kell végrehajtanunk, amivel a relációs sémákat átszervezzük olyan formába, amely veszteségmentes egyesítést eredményez. Ezen normalizálási műveletek sikeres elvégzéséhez a dolgozat két algoritmust is bemutat.

Jelenleg az Újvidéki Egyetem munkatársa vagyok, a Műszaki Tudományok Karán végzek tanársegédi feladatokat. Ezen a karon több szakirányon is folyik relációs adatbázistervezéssel foglalkozó tárgy. A tárgy neve Adatbázisok 2 \parencite{ftn2021}, melynek keretén belül előadjuk az említett normalizálási algoritmusokat is. Az algoritmusokat feladatok kíséretében dolgozzuk fel kézileg és ugyanígy papíron történik a hallgatók vizsgáztatása is. Egy feladatlap elkészítése, megoldáskulcs ellenőrzése, majd a későbbi hallgatók által kitöltött feladatlapok átnézése potenciálisan sok időt felemészt, valamint nagy felelőséggel is jár. A dekompozíció normalizálási algoritmusa választási lehetőség elé állítja az alanyt, vagyis különböző útvonalakon juthat el a hallgató a helyes eredményig. Ez az interaktív mozzanat tovább bonyolítja a megoldott feladatok kiértékelését – adott esetben a részeredmények helyességének a megállapítását.

Annak érdekében, hogy enyhítsünk a tanársegédekre helyezedő nehézségeken, kifejlesztettük a \textit{RelNorm} nevezetű szoftvert, mely képes pontosan elvégezni a relációs adatbázis normalizálási feladatait, lehetővé téve az interaktív lépések végrehajtását. A szoftver emellett még megkönnyíti a feladatsorok megadási módját is, ezzel is felgyorsítja a feladatok kidolgozásának a folyamatát.

A bevezető mondatok után először egy irodalmi áttekintés következik, majd a dolgozat célkitűzéseit definiáljuk. Az ezt követő fejezeteket a szoftver kifejlesztésének elméleti háttere, relációs alapfogalmak és algoritmusok bemutatása követi. A bemutatott algoritmusok megvalósításáról lesz szó az ezt követő fejezetben, majd későbbi fejezetekben a szoftvertesztelés és annak eredményeinek az elemzése kap helyet. Végül záró fejezetként összefoglalásra kerülnek a dolgozatban leírtak.
