\chapter{Irodalom áttekintés}

A bevezető részben már szó volt a relációs adatmodell történetéről, és arról a tényről is, hogy már a múlt század második felében sok kutatói munka tárgyát képezte. Átfogó és bizonyított elméleti alapokkal vághatunk tehát neki a téma kidolgozásának. Ebben a dolgozatban az újvidéki Műszaki Tudományok Karán is tekintélyes irodalomra támaszkodunk \parencite{mogin1996}, \parencite{mogin2004}, \parencite{celikovic2021} és \parencite{kordic2018}. A magyar szakterületen a Budapesti Műszaki Egyetem profeszorának a könyvét \parencite{gajdos2019} idézem a dolgozatom több pontján. A magyar és szerb irodalom nagy hányadában átfedik egymást, ez is a többévtizedes kiforrt és megszilárdult alapoknak köszönhető.

Több kutatás is foglalkozott az évek során az adatbázisok normalizációjának a problémájával. A kutatásaink során kifejezetten ügyeltünk arra, hogy olyan irodalmat szemlézzünk, amelyek gyújtópontjában az oktatás áll.

Antonia Mitrovic munkájában \parencite{mitrovic2002} egy hallgató-központú relációs adatmodellel foglalkozó weboldallal foglalkozik, amely -- többek közt relációs adatbázisok normalizációjáról is szóló -- kérdések-válaszok formájában próbálja meg átadni a tudást a hallgatóknak.

Hongbo Du és Laurent Wery 1999-es dolgozatában \parencite{hongbo1999} foglalkoznak konkrét adatbázis normalizációs problémamegoldó eszközzel. Ez a szoftver grafikus felhasználói felülettel is rendelkezik, mely az 1990-es évek esztétikai világát nyújtja. Feladatok beolvasása nehézkesnek tűnik, mivel több dialóguson keresztül lehet csak megadni függőségeket, ami időigényes feladat. Nincs lehetőség algoritmus kiválasztására a normalizációs probléma megoldására. Kimenetnek a \textit{Microsoft Office} szoftvercsomag \textit{Access} nevezetű adatbáziskezelő programát használja, melyben létrehozza a normalizáció eredményeként létrejött relációs sémákat. Ez bizonyos felhasználási esetekben nagyon is kívánatos eljárás, bár a mi esetünkben egyszerűbb kimeneti eredményt is elfogadhatónak tartanánk.

Amir Bahmani és munkatársai egy automatikus normalizációs eszközről írnak dolgozatukban \parencite{bahmani2008}, mely táblázatok kész adatai alapján határoz meg függőségeket és azok alapján generál normalizált relációs sémákat. A mi elvárásaink alapján a bemeneti adatok helyett elegendő volna csak egy attribútumhalmaz és egy függőséghalmaz betáplálása a programba.

Egy kidolgozott normalizációs eszközről ír Nikolay Georgiev a dolgozatában \parencite{georgiev2008}. Az eszköz rendelkezik grafikus felhasználói felülettel, többdialógusos beviteli lehetőségekkel (melyek némileg lassíthatják a bevitel sebességét) valamint csak egy algoritmus áll rendelkezésre. A szoftver leírásának alapján a szerző inkább a hallgató szemszögéből vezette le a funkcionális követelményeket, így inkább a normalizáció elsajátításán van a szoftverben nagyobb hangsúly, mintsem a feladatsorok gyors átnézésében és megoldásában.
