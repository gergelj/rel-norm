\chapter{Eredmények}

Az előző fejezetben leírt algoritmusok megvalósításának a helyességét úgy tudjuk leellenőrizni, ha szoftvertesztekkel támasztjuk alá a működésüket. Ezeknek a teszteknek az eredményei a ~\ref{tab:teszt} táblázatban szerepelnek. Valamennyi algoritmus metódusához rendeltünk legalább 2 tesztet, hogy megbizonyosodjunk arról, hogy a tesztek nem csak véletlenszerűen sikerültek. A teszt eseteket feladatlapból \parencite{kordic2018} szerzett feladatokra alapoztuk, melyeknek ismertek az eredményei. Ezek az ismert eredmények kerültek összehasonlításra az algoritmus által kapott eredményekkel. Amennyiben az eredmények nem egyeztek meg, azokat sikertelen teszteknek vettük; ha hiba keletkezett tesztelés során, azok a hibák száma oszlopban vannak feltüntetve. Ha bármiféle okból kifolyólag nem lehetett lefuttatni a tesztet, az az átugrott tesztek számát növelte. A tesztek mellett feltüntetett eltelt idő az egyes tesztcsoportok lefutásának az idejét jegyzi.

\begin{table}
    \centering
    \begin{tabular}{|b{4cm}|b{1.5cm}|b{1.8cm}|b{1.5cm}|b{1.5cm}|b{1.5cm}|}
    \hline
    Teszt neve & Tesztek száma & Sikertelen tesztek száma & Hibák száma & Átugrott tesztek száma & Eltelt idő [ms] \\
    \hline \hline
    LogicalConsequence & 5 & 0 & 0 & 0 & 2 \\ \hline
    AttributeSetClosure & 6 & 0 & 0 & 0 & 3 \\ \hline
    Equivalence & 2 & 0 & 0 & 0 & 1 \\ \hline
    Projection & 3 & 0 & 0 & 0 & 1 \\ \hline
    Keys & 3 & 0 & 0 & 0 & 5 \\ \hline
    RelationParser & 2 & 0 & 0 & 0 & 66 \\ \hline
    2NF & 5 & 0 & 0 & 0 & 25 \\ \hline
    3NF & 2 & 0 & 0 & 0 & 1 \\ \hline
    NormalForm & 4 & 0 & 0 & 0 & 5 \\ \hline
    CanonicalSet & 4 & 0 & 0 & 0 & 39 \\ \hline
    \textbf{Decomposition} & 4 & 0 & 0 & 0 & 97 \\ \hline
    \textbf{Synthesis} & 6 & 0 & 0 & 0 & 108 \\ \hline
    \hline
    TOTAL & 46 & 0 & 0 & 0 & 353 \\ \hline
    \end{tabular}
    \caption{Szoftvertesztelés eredménye}
    \label{tab:teszt}
\end{table}

A \textit{SonarCloud} szoftverelemző eszköz jelentésének egy részét a ~\ref{tab:sonar} táblázat jeleníti meg. Ez a jelentés magában foglal különböző statisztikai adatokat a teljes kódbázisról, amit a \textit{SonarCloud} bizonyos belső algoritmusok és kritériumok alapján számol.

\begin{table}
    \centering
    \begin{tabular}{|l|r|}
    \hline
    Jellemző & Érték \\ \hline
    \hline
    Összes kódsor & 1161 \\ \hline
    Osztályok száma & 21 \\ \hline
    Kommentek aránya & 1.4 \\ \hline
    Code Smell & 30 \\ \hline
    Biztonsági rések & 0 \\ \hline
    Kódtöbbszöröződés & 0\% \\ \hline
    Hibák & 0 \\ \hline
    Teszt lefedettség & 76.5\% \\ \hline
    \end{tabular}
    \caption{SonarCloud elemzésének az eredményei}
    \label{tab:sonar}
\end{table}

