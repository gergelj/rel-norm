\selectlanguage{hungarian}
\begin{abstract}

A relációs adatbázisok használata és a velük járó normalizációs problémák már több évtizede foglalkoztatják a számítástechnikában tevékenykedő egyéneket. Az évek során megannyi megoldás született ezekre a problémákra, azok bizonyos aspektusainál mégis hiányosságokat fedezhetünk fel. Az újvidéki Műszaki Tudományok Karán dolgozó tanársegédként egy testreszabott eszköz kifejlesztésében gondolkodtunk, amely kézenfekvő lenne tanársegédi feladatok elvégzéséhez. Fő követelménynek tűztük ki azt, hogy a szoftver támogassa az egyetemen feldolgozott normalizálási algoritmusokat: a szintézis és a dekompozíció algoritmusát. A dekompozíció algoritmusánál további követelmény az algoritmus interaktív lefolyásának a megvalósítása. További részletkövetelmény volt, hogy a feladatsorok könnyen megadhatóak és cserélhetőek legyenek.

Ezen követelmények hatására fejlesztettük ki a \textit{RelNorm} szoftvert, ami a relációs adatbázisok normalizálási problémáját hivatott megoldani konzol applikációként. \textit{Java} programnyelvben íródott applikációról van szó, amely egy relációs sémát bont fel több relációs sémára, a normalizálási algoritmustól függően. Unit-tesztekkel ellenőriztük az algoritmusok helyes megvalósítását, valamint teljes kódbázis elemzést is végrehajtottunk a \textit{SonarCloud} eszközzel. A tesztek eredményét és az elemzést is a dolgozat tartalmazza.

A \textit{RelNorm} a gyakorlatban már bizonyított, ugyanis a 2021--2022-es tanévben már használták a tanársegédek feladatsorok összeállításához és a megoldott feladatlapok átnézésénél.

Kulcsszavak: relációs adatbázis, normalizálás, relációs sémák, szoftvertesztelés, kód lefedettség

\end{abstract}