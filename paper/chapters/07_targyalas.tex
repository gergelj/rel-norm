\chapter{Tárgyalás}

Az előző fejezet szoftvertesztelési eredményei alapján kijelenthető, hogy a \textit{RelNorm} eszköz sikeresen el tudja végezni a relációs adatbázisok normalizációját a szintézis és a dekompozíció algoritmusát felhasználva. Ezek az algoritmusok relatív több időt igényelnek, mint a kisebb összetettségű algoritmusok, és ez a teszteredményeken is meglátszik. A \lstinline{RelationParser} teszt esetek ugornak ki a sorból a hosszabb futtatási idővel. Ezekben az esetekben fájl beolvasása történik, ezzel magyarázható a hosszabb futtatási idő. A dekompozíció algoritmusánál nem sikerült tesztelni az interaktív függőség megadás módszerét, ezért az interaktív módszer azon pontján, ahol felhasználói bemenet szükséges, automatikusan kiválasztja a program a felkínált függőségek egyikét.

A \textit{SonarCloud} jelentése nem okozott kiugró értékeket. Megemlíthető azonban a \textit{Code Smell} jellemző, ami a nem tiszta kód jellemzőit számlálja. Ez az érték 30 lett, ami egy relatív magas szám. Ennek a 30 észrevételnek túlnyomó többsége a felhasználóval való interakciót jelezte, pontosabban a standard bemenet és kimenet használatát kifogásolta. Mivel a \textit{RelNorm} egy konzol applikáció, ezért egyelőre nincs más módja a felhasználói bemenet megváltoztatására.

A teszt lefedettség 76.5\% lett, ami szintén egy jó eredmény. Betekintve az elemzés részleteibe, a lefedettség azért nem lett nagyobb, mert nem került tesztelésre valamennyi konstruktőr, get/set metódus, beépített metódus, valamint a standard bemenettel és kimenettel foglalkozó metódusok sem.
